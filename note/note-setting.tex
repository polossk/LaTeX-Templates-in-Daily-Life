% pdf 设置区
%---------------------------------纸张大小设置---------------------------------%
\usepackage{geometry}
\geometry{a4paper,left=1.91cm,right=1.91cm,top=2.05cm,bottom=2.05cm}
%------------------------------------------------------------------------------%


%----------------------------------必要库支持----------------------------------%
\usepackage{xcolor}
\usepackage{tikz}
\usepackage{layouts}
\usepackage[numbers,sort&compress]{natbib}
\usepackage{clrscode}
\usepackage{gensymb}
%------------------------------------------------------------------------------%


%--------------------------------添加书签超链接--------------------------------%
\usepackage[unicode=true,colorlinks=false,pdfborder={0 0 0}]{hyperref}
% 在此处修改打开文件操作
\hypersetup{
    bookmarks=true,         % show bookmarks bar?
    bookmarksopen=true,     % expanded all bookmark?
    pdftoolbar=true,        % show Acrobat’s toolbar?
    pdfmenubar=true,        % show Acrobat’s menu?
    pdffitwindow=true,      % window fit to page when opened
    pdfstartview={FitH},    % fits the width of the page to the window
    pdfnewwindow=true,      % links in new PDF window
}
% 在此处添加文章基础信息
\hypersetup{
    pdftitle={},
    pdfauthor={polossk},
    pdfsubject={},
    pdfcreator={XeLaTeX},
    pdfproducer={XeLaTeX},
    pdfkeywords={},
}
%------------------------------------------------------------------------------%


%---------------------------------设置字体大小---------------------------------%
\usepackage{type1cm}
% 维持默认行距与字号大小设置
%------------------------------------------------------------------------------%


%---------------------------------设置中文字体---------------------------------%
\usepackage{fontspec}
\usepackage[SlantFont,BoldFont,CJKchecksingle]{xeCJK}
\usepackage{CJKnumb}
% 使用 Adobe 字体
\newcommand\adobeSog{Adobe Song Std}
\newcommand\adobeHei{Adobe Heiti Std}
\newcommand\adobeKai{Adobe Kaiti Std}
\newcommand\adobeFag{Adobe Fangsong Std}
\newcommand\codeFont{Consolas}
% 设置字体
\defaultfontfeatures{Mapping=tex-text}
\setCJKmainfont[ItalicFont=\adobeKai, BoldFont=\adobeHei]{\adobeSog}
\setCJKsansfont[ItalicFont=\adobeKai, BoldFont=\adobeHei]{\adobeSog}
\setCJKmonofont{\codeFont}
\setmonofont{\codeFont}
% 设置字体族
\setCJKfamilyfont{song}{\adobeSog}      % 宋体
\setCJKfamilyfont{hei}{\adobeHei}       % 黑体
\setCJKfamilyfont{kai}{\adobeKai}       % 楷体
\setCJKfamilyfont{fang}{\adobeFag}      % 仿宋体
% 用于页眉学校名,特殊字体,powerby https://github.com/ecomfe/fonteditor
\setCJKfamilyfont{nwpu}{nwpuname}
% 新建字体命令,统一前缀f(a.k.a font)
\newcommand{\fSong}{\CJKfamily{song}}
\newcommand{\fHei}{\CJKfamily{hei}}
\newcommand{\fFang}{\CJKfamily{fang}}
\newcommand{\fKai}{\CJKfamily{kai}}
\newcommand{\fNWPU}{\CJKfamily{nwpu}}
%------------------------------------------------------------------------------%


%------------------------------添加插图与表格控制------------------------------%
\usepackage{graphicx}
\usepackage[font=small,labelsep=quad]{caption}
\usepackage{wrapfig}
\usepackage{multirow,makecell}
\usepackage{longtable}
\usepackage{booktabs}
\usepackage{tabularx}
\usepackage{setspace}
\captionsetup[table]{labelfont=bf,textfont=bf}
\renewcommand\arraystretch{1.1} % 表格行距
%------------------------------------------------------------------------------%


%---------------------------------添加列表控制---------------------------------%
\usepackage{enumerate}
\usepackage{enumitem}
%------------------------------------------------------------------------------%


%---------------------------------设置引用格式---------------------------------%
\renewcommand\figureautorefname{图}
\renewcommand\tableautorefname{表}
\renewcommand\equationautorefname{式}
\newcommand\myreference[1]{[\ref{#1}]}
\newcommand\eqrefe[1]{式(\ref{#1})}
% 增加 \ucite 命令使显示的引用为上标形式
\newcommand{\ucite}[1]{$^{\mbox{\scriptsize \cite{#1}}}$}
\renewcommand{\thefigure}{\thesection-\arabic{figure}}
\renewcommand{\thetable}{\thesection-\arabic{table}}
%------------------------------------------------------------------------------%


%--------------------------------设置定理类环境--------------------------------%
\usepackage[amsthm,thmmarks]{ntheorem}
\newtheorem{thm}{定理}
\newtheorem{prop}[thm]{命题}
\newtheorem{lemma}{引理}
\newtheorem{definition}{定义}
\newtheorem*{solution}{解}
\newtheorem*{solproof}{证明}
\renewcommand{\qed}{\ifmmode\qed\else\hfill\proofSymbol\fi}
%------------------------------------------------------------------------------%


%--------------------------设置中文段落缩进与正文版式--------------------------%
\XeTeXlinebreaklocale "zh"                      % 使用中文的换行风格
\XeTeXlinebreakskip = 0pt plus 1pt              % 调整换行逻辑的弹性大小
\usepackage{indentfirst}                        % 段首空格设置
\setlength{\parindent}{2em}                     % 段首空格长度
% \setlength{\parskip}{3pt plus 1pt minus 1pt}    % 段落间距
\renewcommand{\baselinestretch}{1.25}           % 行距
%------------------------------------------------------------------------------%

%---------------------------------设置页眉页脚---------------------------------%
\usepackage{fancyhdr}
\usepackage{fancyref}
\pagestyle{fancy}
\chead{{\fNWPU 西北工业大学}}
\lhead{}
\rhead{}
\lfoot{}
\cfoot{\thepage}
\rfoot{}
\renewcommand{\headrulewidth}{0.4pt}
\renewcommand{\headwidth}{\textwidth}
\renewcommand{\footrulewidth}{0pt}
%------------------------------------------------------------------------------%


%-------------------------------数学特殊符号控制-------------------------------%
\usepackage{math-symbols}
\numberwithin{equation}{section}
\renewcommand\theequation{\thesection.\arabic{equation}}
%------------------------------------------------------------------------------%


%----------------------------------添加代码控制--------------------------------%
\definecolor{lightgray}{HTML}{cccccc} % \colorlet{lightgray}{gray!40}
\definecolor{darkred}{HTML}{b20000} % \colorlet{darkred}{red!70!black}
\usepackage{listings}
\lstset{
    basicstyle=\color{darkred}\normalsize\ttfamily,
    backgroundcolor=\color{lightgray},
    breaklines=true,
}
%------------------------------------------------------------------------------%

\endinput
%这是简单的 article 的导言区设置,不能单独编译。