\documentclass{article}
% pdf 设置区
%---------------------------------纸张大小设置---------------------------------%
\usepackage{geometry}
\geometry{a4paper,left=1.91cm,right=1.91cm,top=2.05cm,bottom=2.05cm}
%------------------------------------------------------------------------------%


%----------------------------------必要库支持----------------------------------%
\usepackage{xcolor}
\usepackage{tikz}
\usepackage{layouts}
\usepackage[numbers,sort&compress]{natbib}
\usepackage{clrscode}
\usepackage{gensymb}
%------------------------------------------------------------------------------%


%--------------------------------添加书签超链接--------------------------------%
\usepackage[unicode=true,colorlinks=false,pdfborder={0 0 0}]{hyperref}
% 在此处修改打开文件操作
\hypersetup{
    bookmarks=true,         % show bookmarks bar?
    bookmarksopen=true,     % expanded all bookmark?
    pdftoolbar=true,        % show Acrobat’s toolbar?
    pdfmenubar=true,        % show Acrobat’s menu?
    pdffitwindow=true,      % window fit to page when opened
    pdfstartview={FitH},    % fits the width of the page to the window
    pdfnewwindow=true,      % links in new PDF window
}
% 在此处添加文章基础信息
\hypersetup{
    pdftitle={},
    pdfauthor={polossk},
    pdfsubject={},
    pdfcreator={XeLaTeX},
    pdfproducer={XeLaTeX},
    pdfkeywords={},
}
%------------------------------------------------------------------------------%


%---------------------------------设置字体大小---------------------------------%
\usepackage{type1cm}
% 维持默认行距与字号大小设置
%------------------------------------------------------------------------------%


%---------------------------------设置中文字体---------------------------------%
\usepackage{fontspec}
\usepackage[SlantFont,BoldFont,CJKchecksingle]{xeCJK}
\usepackage{CJKnumb}
% 使用 Adobe 字体
\newcommand\adobeSog{Adobe Song Std}
\newcommand\adobeHei{Adobe Heiti Std}
\newcommand\adobeKai{Adobe Kaiti Std}
\newcommand\adobeFag{Adobe Fangsong Std}
\newcommand\codeFont{Consolas}
% 设置字体
\defaultfontfeatures{Mapping=tex-text}
\setCJKmainfont[ItalicFont=\adobeKai, BoldFont=\adobeHei]{\adobeSog}
\setCJKsansfont[ItalicFont=\adobeKai, BoldFont=\adobeHei]{\adobeSog}
\setCJKmonofont{\codeFont}
\setmonofont{\codeFont}
% 设置字体族
\setCJKfamilyfont{song}{\adobeSog}      % 宋体
\setCJKfamilyfont{hei}{\adobeHei}       % 黑体
\setCJKfamilyfont{kai}{\adobeKai}       % 楷体
\setCJKfamilyfont{fang}{\adobeFag}      % 仿宋体
% 用于页眉学校名,特殊字体,powerby https://github.com/ecomfe/fonteditor
\setCJKfamilyfont{nwpu}{nwpuname}
% 新建字体命令,统一前缀f(a.k.a font)
\newcommand{\fSong}{\CJKfamily{song}}
\newcommand{\fHei}{\CJKfamily{hei}}
\newcommand{\fFang}{\CJKfamily{fang}}
\newcommand{\fKai}{\CJKfamily{kai}}
\newcommand{\fNWPU}{\CJKfamily{nwpu}}
%------------------------------------------------------------------------------%


%------------------------------添加插图与表格控制------------------------------%
\usepackage{graphicx}
\usepackage[font=small,labelsep=quad]{caption}
\usepackage{wrapfig}
\usepackage{multirow,makecell}
\usepackage{longtable}
\usepackage{booktabs}
\usepackage{tabularx}
\usepackage{setspace}
\captionsetup[table]{labelfont=bf,textfont=bf}
\renewcommand\arraystretch{1.1} % 表格行距
%------------------------------------------------------------------------------%


%---------------------------------添加列表控制---------------------------------%
\usepackage{enumerate}
\usepackage{enumitem}
%------------------------------------------------------------------------------%


%---------------------------------设置引用格式---------------------------------%
\renewcommand\figureautorefname{图}
\renewcommand\tableautorefname{表}
\renewcommand\equationautorefname{式}
\newcommand\myreference[1]{[\ref{#1}]}
\newcommand\eqrefe[1]{式(\ref{#1})}
% 增加 \ucite 命令使显示的引用为上标形式
\newcommand{\ucite}[1]{$^{\mbox{\scriptsize \cite{#1}}}$}
\renewcommand{\thefigure}{\thesection-\arabic{figure}}
\renewcommand{\thetable}{\thesection-\arabic{table}}
%------------------------------------------------------------------------------%


%--------------------------------设置定理类环境--------------------------------%
\usepackage[amsthm,thmmarks]{ntheorem}
\newtheorem{thm}{定理}
\newtheorem{prop}[thm]{命题}
\newtheorem{lemma}{引理}
\newtheorem{definition}{定义}
\newtheorem*{solution}{解}
\newtheorem*{solproof}{证明}
\renewcommand{\qed}{\ifmmode\qed\else\hfill\proofSymbol\fi}
%------------------------------------------------------------------------------%


%--------------------------设置中文段落缩进与正文版式--------------------------%
\XeTeXlinebreaklocale "zh"                      % 使用中文的换行风格
\XeTeXlinebreakskip = 0pt plus 1pt              % 调整换行逻辑的弹性大小
\usepackage{indentfirst}                        % 段首空格设置
\setlength{\parindent}{2em}                     % 段首空格长度
% \setlength{\parskip}{3pt plus 1pt minus 1pt}    % 段落间距
\renewcommand{\baselinestretch}{1.25}           % 行距
%------------------------------------------------------------------------------%

%---------------------------------设置页眉页脚---------------------------------%
\usepackage{fancyhdr}
\usepackage{fancyref}
\pagestyle{fancy}
\chead{{\fNWPU 西北工业大学}}
\lhead{}
\rhead{}
\lfoot{}
\cfoot{\thepage}
\rfoot{}
\renewcommand{\headrulewidth}{0.4pt}
\renewcommand{\headwidth}{\textwidth}
\renewcommand{\footrulewidth}{0pt}
%------------------------------------------------------------------------------%


%-------------------------------数学特殊符号控制-------------------------------%
\usepackage{math-symbols}
\numberwithin{equation}{section}
\renewcommand\theequation{\thesection.\arabic{equation}}
%------------------------------------------------------------------------------%


%----------------------------------添加代码控制--------------------------------%
\definecolor{lightgray}{HTML}{cccccc} % \colorlet{lightgray}{gray!40}
\definecolor{darkred}{HTML}{b20000} % \colorlet{darkred}{red!70!black}
\usepackage{listings}
\lstset{
    basicstyle=\color{darkred}\normalsize\ttfamily,
    backgroundcolor=\color{lightgray},
    breaklines=true,
}
%------------------------------------------------------------------------------%

\endinput
%这是简单的 article 的导言区设置,不能单独编译。
%------------------------------------------------------------------------------%
\title{张量基础}
\author{polossk}
\date{\today}
\newcommand\artversion{1.0}
%------------------------------------------------------------------------------%
\begin{document}
%------------------------------------------------------------------------------%
\maketitle
\small\tableofcontents
\thispagestyle{fancy}
\renewcommand{\baselinestretch}{1.25}
\fSong\normalsize
%------------------------------------------------------------------------------%
\nocite{MathSymbolsinLaTeXbypolossk}

%------------------------------------------------------------------------------%
\section{符号说明}
%------------------------------------------------------------------------------%
为不失一般性,一般的用如下记号表示标量、向量、张量。

\begin{longtable}[]{@{}llll@{}}
    \toprule
    名称                    & 符号                  & \LaTeX 代码                                                            & 示例      \\
    \midrule
    标量                    & 小写字母              & \lstinline`a`, \lstinline`b`, \lstinline`c` & $a$, $b$,
    $c$                                                                                                                                  \\
    向量                    & \textbf{加粗}小写字母 &
    \lstinline`\mva`,
    \lstinline`\mvb`,
    \lstinline`\mvc`  & $\mva$,
    $\mvb$, $\mvc$                                                                                                                       \\
    向量分量                & 小写字母加下标        & \lstinline`a_i`, \lstinline`b_j`, \lstinline`c_k`
                            & $a_i$, $b_j$, $c_k$                                                                                        \\
    张量                    & \textbf{加粗}大写字母 &
    \lstinline`\mma`,
    \lstinline`\mmb`,
    \lstinline`\mmc` & ${\bm A}$,
    ${\bm B}$, ${\bm C}$                                                                                                                 \\
    张量分量                & 大写字母加下标        & \lstinline`A_{ij}`, \lstinline`B_{ijk}`,
    \lstinline`C_{ijkl}` & $A_{ij}$, $B_{ijk}$,
    $C_{ijkl}$                                                                                                                           \\
    \bottomrule
\end{longtable}
%------------------------------------------------------------------------------%


%------------------------------------------------------------------------------%
\subsection{爱因斯坦记号}
%------------------------------------------------------------------------------%
一般的,如果没有额外说明,在用下标表示的张量计算中,将采用爱因斯坦求和约定(Einstein
summation convention),或者也被称为爱因斯坦记号(Einstein
Notation)。举例说明,如向量的点积(内积),其张量(向量)表示为
$c = \mva \cdot \mvb$,而使用爱因斯坦记号可以表示为
$c = a_i b_i$。这是由于有如下约定

\[a_i b_i = \sum_i a_i b_i = \mva \cdot \mvb\]
这里的下标 $i$ 被称作哑指标,表示对下标 $i$
进行缩并。在爱因斯坦记号的帮助下,许多张量计算可以更简洁的表示。

这里引入两个常用记号,克罗内克 $\delta$ 函数(Kronecker
delta),以及利威尔-奇维塔符号(Levi-Civita symbol)。
%------------------------------------------------------------------------------%


%------------------------------------------------------------------------------%
\subsection{克罗内克 $\delta$ 函数(Kronecker delta)}
%------------------------------------------------------------------------------%
克罗内克 $\delta$ 函数一般被定义为一个二元函数
$\delta_{ij}$,其自变量一般为两个整数,当且仅当两个整数恰好相同时,其取值为
1,否则为 0。即

\[\delta_{ij} = \begin{cases}
        1, & \text{if}\ i = j, \\
        0, & \text{otherwise.}
    \end{cases}\]
%------------------------------------------------------------------------------%


%------------------------------------------------------------------------------%
\subsection{利威尔-奇维塔符号(Levi-Civita symbol)}
%------------------------------------------------------------------------------%
利威尔-奇维塔符号 $\epsilon_{a_1, a_2, \cdots, a_n}$
由其每一个下标指标 $a_1, a_2, \cdots, a_n$
的取值所构成的排列的奇偶性来确定。这里直接给出一些结论:

在二维中,符号 $\epsilon_{ij}$ 的定义如下所示

\[\epsilon_{ij} = \begin{cases}
        +1, & \text{if}\ (i, j) = (1, 2), \\
        -1, & \text{if}\ (i, j) = (2, 1), \\
        0,  & \text{otherwise.}
    \end{cases}\]
此时,$\epsilon_{ij}$ 恰好为一个大小为 $2\times2$ 的反对称矩阵

\[
    \begin{pmatrix}
        \epsilon_{11} & \epsilon_{12} \\ \epsilon_{21} & \epsilon_{22}
    \end{pmatrix}
    = \begin{pmatrix}
        0 & 1 \\ -1 & 0
    \end{pmatrix}
\]
一般的,利威尔-奇维塔符号多见于 3 维或更高的维度当中。以三维的符号
$\epsilon_{ijk}$ 举例,其定义与二维类似

\[\epsilon_{ijk} = \begin{cases}
        +1, & \text{if}\ (i, j, k) \in \bigl\{(1, 2, 3), (2, 3, 1), (3, 1, 2)\bigr\}, \\
        -1, & \text{if}\ (i, j, k) \in \bigl\{(3, 2, 1), (1, 3, 2), (2, 1, 3)\bigr\}, \\
        0,  & \text{otherwise.}
    \end{cases}\]
即其定义是通过序列 $(i, j, k)$
是奇排列还是偶排列定义其取值。一般规定,自然排列 $(1, 2, 3)$ 的取值为
1;当有任意两个指标相同时,其符号为 0。对于序列 $(i, j, k)$
如果可以通过奇数次交换变换成自然排列
$(1, 2, 3)$,则称其为奇排列,其符号取值为
-1;否则为偶排列,其符号取值与自然排列保持一致,为
1。如果将序列扩展到更高的维度,则可以类似的定义符号,这里不再额外赘述。
%------------------------------------------------------------------------------%


%------------------------------------------------------------------------------%
\section{常见运算表}
%------------------------------------------------------------------------------%
下面列举部分简单二元运算的张量写法与爱因斯坦记号写法。为不失一般性,下表中的向量
$\mva \in {\mathbb R}^{n}$ 也可以用列向量
$\mva \in {\mathbb R}^{n \times 1}$
来表示。同时如没有额外说明,张量一般为二阶张量
${\bm A} \in {\mathbb R}^{n \times m}$,同时这里的乘法与指标缩并均满足其定义规定。

\begin{longtable}[]{@{}lll@{}}
    \toprule
    运算                                & 张量写法                                      & 爱因斯坦记号写法\tabularnewline
    \midrule
    \endhead
    同阶向量/张量加减                   &
    ${\bm C} = {\bm A} \pm {\bm B}$     &
    $c_{ij} = a_{ij} \pm b_{ij}$\tabularnewline
    标量与向量/张量相乘                 & ${\bm B} = k {\bm A}$                         &
    $b_{ij} = k a_{ij}$\tabularnewline
    对应项相乘(Hadamard product)      &
    ${\bm C} = {\bm A} \circ {\bm B}$   &
    $c_{ij} = a_{ij} b_{ij}$\tabularnewline
    向量点积(内积)                    &
    $c = \mva \cdot \mvb = \mva^{\text T} \mvb$
                                        & $c = a_i b_i$\tabularnewline
    向量叉积(向量积)                  &
    $\mvc = \mva \times \mvb$           &
    $c_i = \epsilon_{ijk} a_j b_k$\tabularnewline
    向量外积(张量积)                  &
    ${\bm C} = \mva \otimes \mvb= \mva \mvb^{\text T}$
                                        & $c_{ij} = a_i b_j$\tabularnewline
    矩阵转置                            & ${\bm C} = {\bm A}^{\text T}$                 &
    $c_{ji} = a_{ij}$\tabularnewline
    矩阵的迹                            & $c=\mathrm{tr} {\bm A}$                       &
    $c = a_{ii}$\tabularnewline
    矩阵乘法                            & ${\bm C} = {\bm A} {\bm B}$                   &
    $c_{ij} = a_{ik} b_{kj}$\tabularnewline
    Frobenius inner product             &
    ${\bm C} = {\bm A} : {\bm B}=\langle{\bm A}, {\bm B}\rangle_F = \mathrm{tr}\left( {\bm A}^{\text H} {\bm B} \right)$
                                        & $c = \overline{a_{ij}} b_{ij}$\tabularnewline
    Kronecker product                   &
    ${\bm C} = {\bm A} \otimes {\bm B}$ &
    $c_{ijkl} = a_{ij} b_{kl}$\tabularnewline
    \bottomrule
\end{longtable}
%------------------------------------------------------------------------------%

\section{部分公式推导}

\subsection{向量相关}

\paragraph{$(a\cdot b)c$}

\[
    \begin{aligned}
        {\bm d}          & = (\mva \cdot \mvb) \mvc = \mva^{\text T} \mvb \mvc              \\
        \implies d_j     & = (a_i b_i) c_j = a_i (b_i c_j) = c_j b_i a_i = (c_j b_i) a_{i1} \\
        \implies {\bm d} & = (\mvb \otimes \mvc)^{\text T} \mva
    \end{aligned}
\]




%------------------------------------------------------------------------------%

\phantomsection
\addcontentsline{toc}{section}{参考文献}
\bibliographystyle{nputhesis}
\bibliography{reference}
\end{document}