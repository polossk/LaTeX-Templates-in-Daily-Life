\documentclass{article}
\input{note-setting}
%------------------------------------------------------------------------------%
\title{张量基础}
\author{polossk}
\date{\today}
\newcommand\artversion{1.0}
%------------------------------------------------------------------------------%
\begin{document}
%------------------------------------------------------------------------------%
\maketitle
\small\tableofcontents
\thispagestyle{fancy}
\renewcommand{\baselinestretch}{1.25}
\fSong\normalsize
%------------------------------------------------------------------------------%
\nocite{MathSymbolsinLaTeXbypolossk}

%------------------------------------------------------------------------------%
\section{符号说明}
%------------------------------------------------------------------------------%
为不失一般性,一般的用如下记号表示标量、向量、张量。

\begin{longtable}[]{@{}llll@{}}
    \toprule
    名称                    & 符号                  & \LaTeX 代码                                                            & 示例      \\
    \midrule
    标量                    & 小写字母              & \lstinline`a`, \lstinline`b`, \lstinline`c` & $a$, $b$,
    $c$                                                                                                                                  \\
    向量                    & \textbf{加粗}小写字母 &
    \lstinline`\mva`,
    \lstinline`\mvb`,
    \lstinline`\mvc`  & $\mva$,
    $\mvb$, $\mvc$                                                                                                                       \\
    向量分量                & 小写字母加下标        & \lstinline`a_i`, \lstinline`b_j`, \lstinline`c_k`
                            & $a_i$, $b_j$, $c_k$                                                                                        \\
    张量                    & \textbf{加粗}大写字母 &
    \lstinline`\mma`,
    \lstinline`\mmb`,
    \lstinline`\mmc` & ${\bm A}$,
    ${\bm B}$, ${\bm C}$                                                                                                                 \\
    张量分量                & 大写字母加下标        & \lstinline`A_{ij}`, \lstinline`B_{ijk}`,
    \lstinline`C_{ijkl}` & $A_{ij}$, $B_{ijk}$,
    $C_{ijkl}$                                                                                                                           \\
    \bottomrule
\end{longtable}
%------------------------------------------------------------------------------%


%------------------------------------------------------------------------------%
\subsection{爱因斯坦记号}
%------------------------------------------------------------------------------%
一般的,如果没有额外说明,在用下标表示的张量计算中,将采用爱因斯坦求和约定(Einstein
summation convention),或者也被称为爱因斯坦记号(Einstein
Notation)。举例说明,如向量的点积(内积),其张量(向量)表示为
$c = \mva \cdot \mvb$,而使用爱因斯坦记号可以表示为
$c = a_i b_i$。这是由于有如下约定

\[a_i b_i = \sum_i a_i b_i = \mva \cdot \mvb\]
这里的下标 $i$ 被称作哑指标,表示对下标 $i$
进行缩并。在爱因斯坦记号的帮助下,许多张量计算可以更简洁的表示。

这里引入两个常用记号,克罗内克 $\delta$ 函数(Kronecker
delta),以及利威尔-奇维塔符号(Levi-Civita symbol)。
%------------------------------------------------------------------------------%


%------------------------------------------------------------------------------%
\subsection{克罗内克 $\delta$ 函数(Kronecker delta)}
%------------------------------------------------------------------------------%
克罗内克 $\delta$ 函数一般被定义为一个二元函数
$\delta_{ij}$,其自变量一般为两个整数,当且仅当两个整数恰好相同时,其取值为
1,否则为 0。即

\[\delta_{ij} = \begin{cases}
        1, & \text{if}\ i = j, \\
        0, & \text{otherwise.}
    \end{cases}\]
%------------------------------------------------------------------------------%


%------------------------------------------------------------------------------%
\subsection{利威尔-奇维塔符号(Levi-Civita symbol)}
%------------------------------------------------------------------------------%
利威尔-奇维塔符号 $\epsilon_{a_1, a_2, \cdots, a_n}$
由其每一个下标指标 $a_1, a_2, \cdots, a_n$
的取值所构成的排列的奇偶性来确定。这里直接给出一些结论:

在二维中,符号 $\epsilon_{ij}$ 的定义如下所示

\[\epsilon_{ij} = \begin{cases}
        +1, & \text{if}\ (i, j) = (1, 2), \\
        -1, & \text{if}\ (i, j) = (2, 1), \\
        0,  & \text{otherwise.}
    \end{cases}\]
此时,$\epsilon_{ij}$ 恰好为一个大小为 $2\times2$ 的反对称矩阵

\[
    \begin{pmatrix}
        \epsilon_{11} & \epsilon_{12} \\ \epsilon_{21} & \epsilon_{22}
    \end{pmatrix}
    = \begin{pmatrix}
        0 & 1 \\ -1 & 0
    \end{pmatrix}
\]
一般的,利威尔-奇维塔符号多见于 3 维或更高的维度当中。以三维的符号
$\epsilon_{ijk}$ 举例,其定义与二维类似

\[\epsilon_{ijk} = \begin{cases}
        +1, & \text{if}\ (i, j, k) \in \bigl\{(1, 2, 3), (2, 3, 1), (3, 1, 2)\bigr\}, \\
        -1, & \text{if}\ (i, j, k) \in \bigl\{(3, 2, 1), (1, 3, 2), (2, 1, 3)\bigr\}, \\
        0,  & \text{otherwise.}
    \end{cases}\]
即其定义是通过序列 $(i, j, k)$
是奇排列还是偶排列定义其取值。一般规定,自然排列 $(1, 2, 3)$ 的取值为
1;当有任意两个指标相同时,其符号为 0。对于序列 $(i, j, k)$
如果可以通过奇数次交换变换成自然排列
$(1, 2, 3)$,则称其为奇排列,其符号取值为
-1;否则为偶排列,其符号取值与自然排列保持一致,为
1。如果将序列扩展到更高的维度,则可以类似的定义符号,这里不再额外赘述。
%------------------------------------------------------------------------------%


%------------------------------------------------------------------------------%
\section{常见运算表}
%------------------------------------------------------------------------------%
下面列举部分简单二元运算的张量写法与爱因斯坦记号写法。为不失一般性,下表中的向量
$\mva \in {\mathbb R}^{n}$ 也可以用列向量
$\mva \in {\mathbb R}^{n \times 1}$
来表示。同时如没有额外说明,张量一般为二阶张量
${\bm A} \in {\mathbb R}^{n \times m}$,同时这里的乘法与指标缩并均满足其定义规定。

\begin{longtable}[]{@{}lll@{}}
    \toprule
    运算                                & 张量写法                                      & 爱因斯坦记号写法\tabularnewline
    \midrule
    \endhead
    同阶向量/张量加减                   &
    ${\bm C} = {\bm A} \pm {\bm B}$     &
    $c_{ij} = a_{ij} \pm b_{ij}$\tabularnewline
    标量与向量/张量相乘                 & ${\bm B} = k {\bm A}$                         &
    $b_{ij} = k a_{ij}$\tabularnewline
    对应项相乘(Hadamard product)      &
    ${\bm C} = {\bm A} \circ {\bm B}$   &
    $c_{ij} = a_{ij} b_{ij}$\tabularnewline
    向量点积(内积)                    &
    $c = \mva \cdot \mvb = \mva^{\text T} \mvb$
                                        & $c = a_i b_i$\tabularnewline
    向量叉积(向量积)                  &
    $\mvc = \mva \times \mvb$           &
    $c_i = \epsilon_{ijk} a_j b_k$\tabularnewline
    向量外积(张量积)                  &
    ${\bm C} = \mva \otimes \mvb= \mva \mvb^{\text T}$
                                        & $c_{ij} = a_i b_j$\tabularnewline
    矩阵转置                            & ${\bm C} = {\bm A}^{\text T}$                 &
    $c_{ji} = a_{ij}$\tabularnewline
    矩阵的迹                            & $c=\mathrm{tr} {\bm A}$                       &
    $c = a_{ii}$\tabularnewline
    矩阵乘法                            & ${\bm C} = {\bm A} {\bm B}$                   &
    $c_{ij} = a_{ik} b_{kj}$\tabularnewline
    Frobenius inner product             &
    ${\bm C} = {\bm A} : {\bm B}=\langle{\bm A}, {\bm B}\rangle_F = \mathrm{tr}\left( {\bm A}^{\text H} {\bm B} \right)$
                                        & $c = \overline{a_{ij}} b_{ij}$\tabularnewline
    Kronecker product                   &
    ${\bm C} = {\bm A} \otimes {\bm B}$ &
    $c_{ijkl} = a_{ij} b_{kl}$\tabularnewline
    \bottomrule
\end{longtable}
%------------------------------------------------------------------------------%

\section{部分公式推导}

\subsection{向量相关}

\paragraph{$(a\cdot b)c$}

\[
    \begin{aligned}
        {\bm d}          & = (\mva \cdot \mvb) \mvc = \mva^{\text T} \mvb \mvc              \\
        \implies d_j     & = (a_i b_i) c_j = a_i (b_i c_j) = c_j b_i a_i = (c_j b_i) a_{i1} \\
        \implies {\bm d} & = (\mvb \otimes \mvc)^{\text T} \mva
    \end{aligned}
\]




%------------------------------------------------------------------------------%

\phantomsection
\addcontentsline{toc}{section}{参考文献}
\bibliographystyle{nputhesis}
\bibliography{reference}
\end{document}