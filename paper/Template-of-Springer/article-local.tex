%!TEX TS-program = xelatex
\documentclass[pdflatex, a4paper, default]{sn-jnl}
% \documentclass[referee]{sn-jnl}
%------------------------------------------------------------------------------%
% To get PDF file: xelatex -> bibtex -> xelatex -> xelatex
%------------------------------------------------------------------------------%
\jyear{2022}

%------------------------------------------------------------------------------%
% Springer Default LaTeX Setting, replace it when submitting your paper
\input{default-sn-journal.tex}
\usepackage{lineno}
%------------------------------------------------------------------------------%


% Math Symbols
\usepackage{math-symbols}
\usepackage{cases}

\newcommand\artversion{1.0} % easy to track and folk
\newcommand{\R}[1]{\textcolor{red}{#1}} % text in red
\newcommand{\RU}[1]{\textcolor{red}{#1}} %
\newcommand{\blu}[1]{\textcolor{blue}{#1}} %
\newcommand{\dto}[1]{\textcolor{OliveGreen}{#1}} %
% added by ditho
\newcommand\eg{{e.g.}\ }
\newcommand\etal{{et al.}\ }
\newcommand\ie{{i.e.}}
\newcommand{\dbquote}[1]{\textquotedblleft #1\textquotedblright}
\newcommand{\sgquote}[1]{\textquoteleft #1\textquoteright}
\graphicspath{{./images/}}
% diff operator
\newcommand\dx{{\diff x}}
\newcommand\dxp{{\diff x'}}
\newcommand\dVmvx{{\diff V_{\mvx}}}
\newcommand\dVmvxp{{\diff V_{\mvx'}}}
% trivial PD
\newcommand\horizondeltax{{\mathcal H_{\delta_0}(x)}}
\newcommand\horizondeltaxp{{\mathcal H_{\delta_0}(x')}}
\newcommand\horizondeltamvx{{\mathcal H_{\delta_0}(\mvx)}}
\newcommand\horizondeltamvxp{{\mathcal H_{\delta_0}(\mvx')}}
% Dual-horizon PD
\newcommand\horizonx{{\mathcal H_{x}}}
\newcommand\horizonmvx{{\mathcal H_{\mvx}}}
\newcommand\horizonmvxp{{\mathcal H_{\mvx'}}}
\newcommand\Dualhorizonx{{\mathcal H'_{x}}}
\newcommand\Dualhorizonmvx{{\mathcal H'_{\mvx}}}
\newcommand\Dualhorizonmvxp{{\mathcal H'_{\mvx'}}}
% boundary
\newcommand\Bdelta{{\mathrm B_\delta}}
\newcommand\Idelta{{\mathrm I_\delta}}
% more readable
\newcommand\Hdeltax{\horizondeltax}
\newcommand\Hdeltaxp{\horizondeltaxp}
\newcommand\Hdeltamvx{\horizondeltamvx}
\newcommand\Hdeltamvxp{\horizondeltamvxp}
\newcommand\Hx{\horizonx}
\newcommand\Hmvx{\horizonmvx}
\newcommand\Hmvxp{\horizonmvxp}
\newcommand\DHx{\Dualhorizonx}
\newcommand\DHmvx{\Dualhorizonmvx}
\newcommand\DHmvxp{\Dualhorizonmvxp}

%--------------------------------------------------------------------------%
\begin{document}
%--------------------------------------------------------------------------%
\title[Your title ABBR]{Your title}

\author*[1]{\fnm{Nide} \sur{Mingzi}}
\email{exampel@mail.domain.com}

\author[1]{\fnm{Zhang} \sur{San}}

\author[1]{\fnm{Li} \sur{Si}}

\author[2]{\fnm{Wang} \sur{Wu}}

\affil*[1]{\orgdiv{School of Mathematics and Statistics},
    \orgname{Northwestern Polytechnical University},
    \orgaddress{\city{Xi'an} \postcode{710072}, \country{China}}}

\affil[2]{\orgdiv{Another School of Mathematics and Statistics},
    \orgname{Northwestern Polytechnical University},
    \orgaddress{\city{Xi'an} \postcode{710072}, \country{China}}}

\abstract{
    A balabala balabala balabala \dots
}

\keywords{
    Word, Word, Word.
}
%------------------------------------------------------------------------------%

\maketitle

%------------------------------------------------------------------------------%
\section*{Version: \artversion, Date: \today} % comment this line for submission
%------------------------------------------------------------------------------%


%------------------------------------------------------------------------------%
\section{Introduction}\label{sec:introduction}
%------------------------------------------------------------------------------%
Long long ago, we \dots \cite{coussy1995mechanics,shen2021peridynamic}
%------------------------------------------------------------------------------%


%------------------------------------------------------------------------------%
\section{Basic knowledge}\label{sec:sec-2}
%------------------------------------------------------------------------------%

%------------------------------------------------------------------------------%


%------------------------------------------------------------------------------%
\section{Models}\label{sec:sec-3}
%------------------------------------------------------------------------------%

%------------------------------------------------------------------------------%


%------------------------------------------------------------------------------%
\section{Algorithm}\label{sec:sec-4}
%------------------------------------------------------------------------------%

%------------------------------------------------------------------------------%


%------------------------------------------------------------------------------%
\section{Numerical Examples}\label{sec:sec-5}
%------------------------------------------------------------------------------%

%------------------------------------------------------------------------------%


%------------------------------------------------------------------------------%
\section{Conclusion}\label{sec:sec-6}
%------------------------------------------------------------------------------%
In this paper, we have \dots
%------------------------------------------------------------------------------%


%------------------------------------------------------------------------------%
\bmhead{Acknowledgements}

\noindent This research was supported by the \dots
%------------------------------------------------------------------------------%
\begin{appendices}
    \section{The Supplementary Material}\label{sec:app-A}
\end{appendices}
%------------------------------------------------------------------------------%
% \section*{Reference}
%------------------------------------------------------------------------------%
\addcontentsline{toc}{section}{Reference}
\markboth{Reference}{}
% \bibliographystyle{sn-basic}
\bibliography{reference}

%------------------------------------------------------------------------------%

\end{document}